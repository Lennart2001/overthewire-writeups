\documentclass[conference]{IEEEtran.cls}
\IEEEoverridecommandlockouts
% The preceding line is only needed to identify funding in the first footnote. If that is unneeded, please comment it out.
\usepackage{cite}
\usepackage{amsmath,amssymb,amsfonts}
\usepackage{graphicx}
\usepackage{textcomp}
\usepackage{xcolor}
\usepackage[hyphens]{url}
\usepackage{hyperref}
\usepackage{listings}
\def\BibTeX{{\rm B\kern-.05em{\sc i\kern-.025em b}\kern-.08em
    T\kern-.1667em\lower.7ex\hbox{E}\kern-.125emX}}

\newcommand{\rom}[1]{\uppercase\expandafter{\romannumeral #1\relax}}

\begin{document}

\title{HackTheBox: WhatIsIt?}

\author{\IEEEauthorblockN{Lennart Buhl}
\IEEEauthorblockA{\textit{Department of Computer Science} \\
\textit{University of St. Thomas}\\
\textit{September, 2023}}}

\maketitle



\begin{abstract}

\end{abstract}



\begin{IEEEkeywords}
programming, cybersecurity, security, pentesting
\end{IEEEkeywords}


\section{Introduction}
Redeemer is located at IPv4 10.129.223.3. Which we can access through the HackTheBox OpenVPN gateway.
We achieve this by simply running this command in a shell:
\begin{scriptsize}
\begin{verbatim}
sudo openvpn starting_point_{user}.ovpn
\end{verbatim}
\end{scriptsize}

\\
\textit{I will now switch to root shell.}
\\

Once we are on the VPN, we can check if we have access to the machine at 10.129.223.3:
\begin{scriptsize}
\begin{verbatim}
root@ghost:~# ping 10.129.223.3

\end{verbatim}
\end{scriptsize}

\begin{thebibliography}{00}

\bibitem{Khomtchouk} B. B. Khomtchouk, “Codon usage bias levels predict taxonomic identity and genetic composition,” bioRxiv, 01-Jan-2020. \url{https://doi.org/10.1101/2020.10.26.356295}.

\end{thebibliography}
\vspace{12pt}


\end{document}